%----------------------------------------------------------------------------------------
%	PACKAGES AND OTHER DOCUMENT CONFIGURATIONS
%----------------------------------------------------------------------------------------
\documentclass[margin, centered]{res}
\usepackage{xeCJK}
\topmargin=-0.5in
\oddsidemargin -.5in
\evensidemargin -.5in
\textwidth=6.5in
\itemsep=0in
\parsep=0in
\newsectionwidth{1in}
\usepackage[pdftex]{graphicx}
\usepackage{enumitem}
\usepackage{wrapfig}
\usepackage{helvet}
\usepackage[colorlinks = true,
    linkcolor = blue,
    urlcolor  = blue,
    citecolor = blue,
    anchorcolor = blue]{hyperref}
\setlength{\textwidth}{6.5in} % Text width of the document
\setlength{\textheight}{720pt}

\begin{document}

%----------------------------------------------------------------------------------------
%	NAME AND ADDRESS SECTION
%----------------------------------------------------------------------------------------\
    \begin{center}
        \hspace{-\hoffset}
        \huge\bf{万豪}
    \end{center}
    \vspace*{5mm}
    \vspace{-7mm}
    \moveleft\hoffset\vbox{\hrule width 19cm height 0.5pt}
    \vspace{-9mm}
    \begin{center}
        \hspace{-\hoffset}
        \href{mailto:2235623837@qq.com}{2235623837@qq.com} ~\textbullet~ \ 15809265710  ~\textbullet~ JAVA研发工程师
    \end{center}
    \vspace{-7mm}
    \begin{resume}

%----------------------------------------------------------------------------------------
%	EDUCATION SECTION
%----------------------------------------------------------------------------------------
        \section{教育}
        \textbf{软件工程} \hfill 2015 - 2019 \\
        { 西北工业大学 \textit{工学学士} }


%----------------------------------------------------------------------------------------
%	TECHNICAL SKILLS SECTION
%----------------------------------------------------------------------------------------
        \section{专业技能}

        \textbf{编程语言} - Java \\
        \textbf{开源框架} - Spring,Dubbo,Zookeeper,Tair,Redis,Metaq,MNS\\
        \textbf{工作领域} - SDN网络应用层 \\


%----------------------------------------------------------------------------------------
%	Selected Projects Section
%----------------------------------------------------------------------------------------
        \section{工作经历}
        阿里云计算有限公司  \hfill 2019年至今


        高速通道  (Express Connect)
        \\
%\setlist[itemize]{
        \begin{itemize}[leftmargin=*]


            \item \textbf{{产品功能 }}: 用户IDC机房接入云上网络。

            \item \textbf{{流量转发}} : 通过阿里接入交换机CSW和用户机房交换机相连,CSW将用户侧流量做vxlan封装后,转发到云上阿里虚拟交换机AVS,即可实现用户机房和云上资源的联通。

        \end{itemize}



        任播弹性公网IP(AnycastEIP)
        \\
%\setlist[itemize]{
        \begin{itemize}[leftmargin=*]


            \item \textbf{{产品功能 }}: 覆盖全球的公网访问加速产品。

            \item \textbf{{流量转发 }}:通过阿里路由器IGW在全球20+个站点宣告BGP路由,用户请求会走就近站点进入IGW,IGW将请求经过内部专线转发到阿里云上服务,通过走专线的方式提高网络质量。

        \end{itemize}


        巡检系统开发
        \\
%\setlist[itemize]{
        \begin{itemize}[leftmargin=*]


            \item \textbf{{项目背景 }}: 多个系统间数据不一致,例如用户看到的配置和网络底层配置不一致,会导致网络产品行为不符合用户预期,不一致数据是系统风险,需要进行对比,及时发现线上风险。

            \item \textbf{{实现概述 }} : 定时从多个只读数据库查询数据并进行对比的方式,来发现和解决多个系统间的数据库数据不一致问题,发现并收敛上百个风险点,提高系统稳定性。

            \item \textbf{{项目难点 }} : 多个系统数据库数据更新时间有时间差,通过多次校验异常数据的方式,减少误报数据。

        \end{itemize}


        本地集成测试系统开发
        \\
%\setlist[itemize]{
        \begin{itemize}[leftmargin=*]


            \item \textbf{{项目背景 }}: 本地构建测试环境困难,需要手动插入sql,代码可复用性弱。测试场景校验DB困难,需要通过sql查询出db内容,再去校验是否符合预期。

            \item \textbf{{实现概述 }} :通过在内存中一键创建JavaBean,支持将JavaBean映射为sql存储数据库,解决构造依赖资源困难的问题。测试结束自动将JavaBean和数据库数据对比,解决了校验数据库操作复杂的问题。提高测试开发效率。

            \item \textbf{{项目难点}} : 性能优化:通过Druid的Filter来记录测试时操作的表,对比和清理表时,去对比操作过的表,而不是捞取全部表来对比和清理。


        \end{itemize}


%----------------------------------------------------------------------------------------
%	RELEVANT COURSE SECTION
%----------------------------------------------------------------------------------------
% \section{Courses}
% Data Structures and Algorithms, Operating System, Computer Graphics, Unix Programming, Machine Learning and Pattern Recognition,Artificial Intelligence,Discrete Mathematics, Advance Computer Networks, Databases, Theory Of Computation and Automata, Software Engineering.

%----------------------------------------------------------------------------------------
% \section{Hackathons}
% \begin{itemize}[leftmargin=*]
%  \item \textbf{{InGENIUS}} 24Hr 2015 : Built an Aptitude Android App "Triviality".
%  \item \textbf{{CODE2k16}} 24 Hr 2016 : Presented the Face Authorization Project.
%  \item \textbf{{init}} 12Hr 2016 : Built a Food Donate Android App "FeedTheNeed".
%  \item \textbf{{KLUDGE}} 24 Hr 2017 : Presented the Accident Prevention Project.
%  \item \textbf{{MANTHAN}}, BIT 2016 : Built an Android App "GreenBen".
%  \item \textbf{{InGENIUS}} 24Hr 2017 : Built a Face Emotion Recognition App.
% \end{itemize}

%----------------------------------------------------------------------------------------
%	ACHIEVEMENT SECTION
%----------------------------------------------------------------------------------------

% \section{Achievements and Awards}
% \begin{itemize}[leftmargin=*]
%  \item 6th International Olympiad Of Mathematics 2013. (\textbf{{State rank : 110}} )
%  \item Bagged \textbf{{Second}} Prize in 'Debug IT' Intercollegiate Debugging Competition, Nov 2013.
%  \item Selected for \textbf{{Top 5}} amongst 120 Teams in KLUDGE 24-Hour Hackathon, Apr 2017.
%  \item Secured \textbf{{First}} Prize in Senior Category at school level - Titan Genius Kidz National
% Arithmetic Championship 2009.
%  \item Secured \textbf{{7th (Top 10)}} in Computer Science in India - QuizUp, Sept 2016.
% \end{itemize}


%	HOBBIES SECTION
%----------------------------------------------------------------------------------------

% \section{Hobbies}
% Writing on \href{https://www.quora.com/profile/Sharath-Guna-1}{Quora}, Stock and Crypto Trading, Learning Cosmos, Trekking and Reading Classics.

% \section{Strengths}
% An Autodidact , Quick Grasper , Brisk Learner, Acclimatized and Adaptive Person, Good Team-worker.

    \end{resume}
\end{document}
